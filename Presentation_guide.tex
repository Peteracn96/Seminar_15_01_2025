
%%%%%%%%%%%%%%%WHAT IS SCREENING%%%%%%%%%%%%%%%%%%%%
Slide 6

-Here I present the concept of screening, which can appear in different contexts.

-For example, in an atom, we say that the core electrons screen the Coulomb 
attraction between the ion and the outer valence electrons, which see an effective 
nuclear charge smaller than that of the nucleus, given by an effective nuclear number
Zeff.

-In this way, the potential felt by the outer electrons due to the nucleus is a screened 
potential, smaller in magnitude than the initial bare Coulomb potential.

-In a very different context, the case of plasmas, the ionic species attract the not so 
energetic free electrons to gather around them, leading due a diminished positive ionic 
charge. The electrostatic potential created by these charges is then screened with a 
screening length parameter denoted by lambda_D, the Debye length. For further reading
you can consult the textbook by Francis Chen

-In the context relevant to us, we can also speak of screening in crystaline solids,
where the screening is captured by the dielectric function. It's clearly a 2-point
function in space, as the what happens here at this point affects what happens at
this other point. So let's dwell on the dielectric function
%%%%%%%%%%%%%%%THE DIELECTRIC FUNCTION%%%%%%%%%%%%%%
Slide 8

-How can the dielectric function be calculated?

-Many approaches depending on the context, to arrive at the same end results.

-1st, Self-consistent framework
Within a self-consistent framework, one can define an interacting polarizability 
as a response to the external potential, and an independent particle polarizability 
function as a response to the total perturbing potential. Under the approximation of 
neglecting electronic correlations and exchanges, we get a relation between the 
full polarizability and the independent particle one, which leads to the Random 
Phase Approximation dielectric function, which is the one we compute in this work. 
From now on, Random Phase Approximation will go by RPA.

The expression for the independet particle polarizability can be obtained through 
basic time-dependent perturbation theory calculations.

-2nd Many-Body Perturbation Theory

Here the polarizability, the screened potential and the self-energy are written in 
diagrammatic expansions. Upon a resummation of the diagrams for the self-energy, 
where only the most divergent ones are retained, the bubble diagrams, are kept 
in the sum. In this way we obtain the RPA self-energy. Something similar can be 
done for the screened potential, and a resummation of the diagrams for the 
polarizability allow for a simple expression for the dielectric function.
In this context the most used polarizability is the RPA irreducible polarizability,
which coincides with the bubble diagram, also called the Hartree diagram or Hartree term.

-3rd Linear response

The polarizability is the density-density response function, pertains to the response
to longitudinal oscillations of the electric field. There are other response functions,
such that the current-current response function, which leads to the optical conductivity
that pertains to the response to electromagnetic oscillations, as in the previous work of
mine where we computed the excitonic optical response of hBN, but not to be confunded with
the one computed in this work.

All in all, we all compute the same thing at the end. One of the first and most important
works about the dielectric function in solids is this by Stadler.

Slide 9

Before presenting my results, we look at useful expressions to compute the dielectric function. 
